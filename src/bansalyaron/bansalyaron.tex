\documentclass[english]{article}
\usepackage[T1]{fontenc}
\usepackage[latin9]{inputenc}
\usepackage{color}
\usepackage{amsmath}
\usepackage{amssymb}
\usepackage{esint}
\usepackage{babel}
\begin{document}

\section{Derivation}
HJB is
$$0 = \rho \theta V_t(\frac{C_t^{\frac{1-\gamma}{\theta}}}{((1-\gamma)V_t)^{\frac{1}{\theta}}}-1) + \frac{E[dV_t]}{dt}$$
Let's define $G_t$ such that
$$V_t = \frac{C_t^{1-\gamma}}{1-\gamma} G_t$$
By Ito, 
$$0 = \rho\theta(G_t^{1-\frac{1}{\theta}}-G_t)   +  G_t E\frac{\frac{dC_t^{1-\gamma}}{C_t^{1-\gamma}}}{dt} + E\frac{dG_t}{dt} + E\frac{dG_t\frac{dC_t^{1-\gamma}}{C_t^{1-\gamma}}}{dt}$$
Denoting $\mu_{C}$ and $\sigma_{C}$ the geometric drift of $C_t$, we have
\begin{align*}
	\frac{dC_{t}^{1-\gamma}}{C_{t}^{1-\gamma}}=((1-\gamma)\mu_{C}-\frac{1}{2}(1-\gamma)\gamma\sigma^{2}_{C})dt + (1-\gamma)\sigma_{C}dW_t
\end{align*}
Injecting this expression into HJB and denoting $\mu_G, \sigma_G$ the arithmetic drift and volatility of $G_t$
$$0 = \rho \theta (G_t^{1-\frac{1}{\theta}}-G_t)  + G_t ((1-\gamma) \mu_{C} - \frac{1}{2}(1-\gamma)\gamma\sigma_{C}'\sigma_{C}) +  \mu_G + \sigma_G'(1-\gamma)\sigma_{C}$$

\section{Long run risk model}

\subsection{Derivation}
We now assume that the evolution of consumption is driven by two state variables $\mu_{t}$ and $\sigma_{t}$:
\begin{align*}
	\frac{dC_{t}}{C_{t}} & =  \mu_{t}dt+\nu_{D}\sqrt{\sigma_{t}}dZ_{t}\\
	d\mu_{t} & =  \kappa_{\mu}(\bar{\mu}-\mu_{t})dt+\nu_{\mu}\sqrt{\sigma_{t}}dZ_{t}^{\mu}\\
	d\sigma_{t} & =  \kappa_{\sigma}(1-\sigma_{t})dt+\nu_{\sigma}\sqrt{\sigma_{t}}dZ_{t}^{\sigma}
\end{align*}
We write $G_t = G(\mu, \sigma)$ and we get the PDE
\begin{align*}
	0&= \rho \theta[G^{1-\frac{1}{\theta}}- G]+G((1-\gamma)\mu-\frac{1}{2}(1-\gamma)\gamma\nu_D^2\sigma)\\
	&+ \kappa_{\mu}(\bar{\mu}-\mu)\frac{\partial G}{\partial\mu}+  \kappa_{\sigma}(1-\sigma)\frac{\partial G}{\partial\sigma}\\
	&+\frac{1}{2}\nu_{\mu}^{2}\sigma\frac{\partial^{2}G}{\partial\mu^{2}}+\frac{1}{2}\nu_{\sigma}^{2}\sigma \frac{\partial^{2}G}{\partial\sigma^{2}}
\end{align*}

\subsubsection{Finite Difference Method}
We can discretize this PDE on a grid using a Finite Difference Scheme. 
We assume that the processes associated with state variables have reflecting boundaries at the borders (except at the LHS border for $\sigma$). This gives a supplementary condition of the form $\partial G = 0$ at the borders\footnote{At the frontier we have
$$dG_{t} = G'(x) \sigma(x) \sqrt{dt} + G'(x)\mu(x) dt + \frac{1}{2}G''(x)\sigma^2(x)dt + o(dt)$$
Since $dG_t$ is $O(dt)$, we must have $G'(x) = 0$, at least when $\sigma(x) \neq 0$.}


To handle boundary conditions, we upwind the first derivative. This means that we approximate $\partial G$ by a forward difference when the drift is positive and a backward difference when the drift is negative. The main advantage here is that at the frontier, the first derivative does not involve points outside the grid. The second derivative still does. We find the value of $G$ at these points by applying the reflecting boundary condition there \footnote{At the points $j = 0$, the volatility is zero. Although we don't have the border condition $\partial_\sigma G = 0$ or $\partial_\mu G = 0$ anymore, the term involving the second derivative disappears naturally}


To sum up the scheme is 
\begin{align*}
	0&= \rho \theta[(G_{ij})^{1-\frac{1}{\theta}}- G_{ij}]+G_{ij}((1-\gamma)\mu_i-\frac{1}{2}(1-\gamma)\gamma\nu_D^2\sigma_j)\\
	&+(\kappa_{\mu_i}(\bar{\mu}-\mu_i))^+\frac{G_{i+1, j}-G_{i, j}}{\Delta \mu}+(\kappa_{\mu_i}(\bar{\mu}-\mu_i))^-\frac{G_{i, j}-G_{i-1, j}}{\Delta \mu}\\
	&+(\kappa_{\sigma_j}(1-\sigma_j))^+\frac{G_{i, j+1}-G_{i,j}}{\Delta \sigma}+(\kappa_{\sigma_j}(1-\sigma_j))^-\frac{G_{i, j}-G_{i,j-1}}{\Delta \sigma}\\
	&+\frac{1}{2}\nu_{\mu_i}^{2}\sigma_j\frac{G_{i+1, j} - 2 G_{i, j} + G_{i-1, j}}{(\Delta\mu)^2}+\frac{1}{2}\nu_{\sigma_j}^{2}\sigma_j\frac{G_{i, j+1} - 2 G_{i, j} + G_{i, j-1}}{(\Delta\sigma)^2}
\end{align*}

Denote $Y$ the vector of  $(G_{ij})_{1 \leq i,j\leq n}$. The scheme  defines a function $F$ such that $F(Y) = 0$. We can solve for $Y$ using one of these methods:
\begin{enumerate}
	\item Use a non linear solver for the system $F(Y) = 0$
	\item Use an ODE solver for the system $F(Y) = \dot{Y}$. The solution when $T\rightarrow +\infty$ is the solution of the PDE. This method is called ``the method of lines''.
\end{enumerate}
Both methods require the jacobian of $F$, which can be automatically computed using automatic differenciation or numerical differenciation.

A solution that would \textit{not} work is to solve for $G$ by iterating over time
\begin{align*}
	\frac{G_{n+1}-G_n}{\Delta t} &= F(G_n)
\end{align*}
This method can be seen as a special case of a non linear solver (fixed point method) or as a special case of an ODE solver (in this context, it is called the Euler method). The criterion for the convergence of this method is that $F$ is monotonous in $G$ (Barles Souganadis theorem). This is not the case here, due to the non linear term  $\rho \theta[(G_{ij})^{1-\frac{1}{\theta}}- G_{ij}]$
For the initial guess, we use the value function for the stationary problem $\sigma = 1$ and $\mu = \overline{\mu}$

\subsubsection{Spectral Method (= collocation method)}
We can also solve the PDE by looking at solutions of the form
$$V(\mu, \sigma) = \sum_{kl} a_{kl} \phi_k(\mu)\psi_l(\sigma)$$
where $\phi, \psi$ denote any basis of functions (splines or chebyshev polynomials).
The value function is characterized by its coordinates $a_{kl}$ on this basis. Writing the PDE on a grid gives a non linear system in term of the coordinates, which we can solve. We use the same border condition and initial guess as the Finite Difference method.

\subsection{Comparison}

\begin{tabular}{|c|c|c|c|c|c|}
	\hline 
	Name & BY04 & BY04 & This paper & This paper & Link
	\\
	\hline 
	\hline 
	mean growth rate & $\mu$ & 0.0015 & $\bar{\mu}$ & 0.0015 & $\mu=\bar{\mu}$
	\\
	\hline 
	mean volatility & $\sigma^{2}$ & 0.00006084 & $\nu_{D}$ & 0.0078 & $\sqrt{\sigma^{2}}=\nu_{D}$
	\\
	\hline 
	growth persistence & $\rho$ & 0.979 & $\kappa_{\mu}$ & 0.0212 & $ - \log(\rho) = \kappa_\mu$ 
	\\
	\hline 
	volatility persistence & $\nu_{1}$ & 0.987 & $\kappa_{\sigma}$ & 0.0131 & $-\log\left(\nu_{1}\right)=\kappa_{\sigma}$
	\\
	\hline 
	growth rate volatility & $\varphi_{e}$ & 0.044 & $\nu_{\mu}$ & 0.0003432 & $\varphi_{e}\times\sqrt{\sigma^{2}}=\nu_{\mu}$
	\\
	\hline 
	volatility volatility & $\sigma_{w}$ & 0.0000023 & $\nu_{\sigma}$ & 0.0378 & $\sigma_{w}/\sigma^{2}=\nu_{\sigma}$
	\\
	\hline 
	time discount & $\delta$ & 0.998 & $\rho$ & 0.002 & $-\log\left(\delta\right)=\rho$
	\\
	\hline 
	RRA & $1-\gamma$(RRA) & 7.5 or 10 & $1-\gamma$ & -6.5 or -9 & $1-\text{RRA}=1-\gamma$
	\\
	\hline 
	IES & $\psi$ & 1.5 & $\psi$ & 1.5 & $\psi = \psi$
	\\
	\hline
\end{tabular}
Also,  $\theta = (1-\gamma)/(1- 1/\psi)$ = -19.50 or -27.
Let's express the  wealth to consumption ratio $K_t$ in term of state variables.
$$V = G_tK_t^{\gamma - 1}\frac{W^{1-\gamma}}{(1-\gamma)}$$
FOC for consumption can be written
$$K_t^{-1} = \rho^{\psi} K_t^{\psi - 1}G_t^\frac{1-\psi}{1-\gamma}$$
We conclude 
$$K_t = \rho^{-1} G_t^{1/\theta}$$
Bansal Yaron find
\begin{align*}
	\log K_t &\propto A_1 \mu_t + A_2 \nu_D^2\sigma_t\\
	A1 &= \frac{1-\frac{1}{\psi}}{1-0.997 e^{-\kappa_\mu}}\\
	A2 &= 0.5\theta\frac{(1 - \frac{1}{\psi})^2 + (A_1  0.997 \frac{\nu_\mu}{\nu_D})^2}{1-0.997e^{-\kappa_\sigma}}
\end{align*}
\end{document}