\documentclass[english]{article}
\usepackage[T1]{fontenc}
\usepackage[latin9]{inputenc}
\usepackage{color}
\usepackage{amsmath}
\usepackage{amssymb}
\usepackage{esint}
\usepackage{babel}
\begin{document}

\section{Derivation}
HJB is
$$0 = \rho \theta V_t(\frac{C_t^{\frac{1-\gamma}{\theta}}}{((1-\gamma)V_t)^{\frac{1}{\theta}}}-1) + \frac{E[dV_t]}{dt}$$
Let's define $G_t$ such that
$$V_t = \frac{C_t^{1-\gamma}}{1-\gamma} G_t$$
$G$ is directly related to the wealth/consumption ratio:
$$WC_{t} = \frac{G_t^{1/\theta}}{\rho}$$

By Ito, denoting $\mu_{C}$ and $\sigma_{C}$ the geometric drift of $C_t$, and $\mu_G, \sigma_G$ the arithmetic drift and volatility of $G_t$
$$0 = \rho \theta (G_t^{1-\frac{1}{\theta}}-G_t)  + G_t ((1-\gamma) \mu_{C} - \frac{1}{2}(1-\gamma)\gamma\sigma_{C}'\sigma_{C}) +  \mu_G + \sigma_G'(1-\gamma)\sigma_{C}$$


\section{Long run risk model}

\subsection{Derivation}
We now assume that the evolution of consumption is driven by two state variables $\mu_{t}$ and $\sigma_{t}$:
\begin{align*}
	\frac{dC_{t}}{C_{t}} & =  \mu_{t}dt+\nu_{D}\sqrt{\sigma_{t}}dZ_{t}\\
	d\mu_{t} & =  \kappa_{\mu}(\bar{\mu}-\mu_{t})dt+\nu_{\mu}\sqrt{\sigma_{t}}dZ_{t}^{\mu}\\
	d\sigma_{t} & =  \kappa_{\sigma}(1-\sigma_{t})dt+\nu_{\sigma}\sqrt{\sigma_{t}}dZ_{t}^{\sigma}
\end{align*}
We write $G_t = G(\mu, \sigma)$ and we get the PDE
\begin{align*}
	0&= \rho \theta[G^{1-\frac{1}{\theta}}- G]+G((1-\gamma)\mu-\frac{1}{2}(1-\gamma)\gamma\nu_D^2\sigma)\\
	&+ \kappa_{\mu}(\bar{\mu}-\mu)\frac{\partial G}{\partial\mu}+  \kappa_{\sigma}(1-\sigma)\frac{\partial G}{\partial\sigma}\\
	&+\frac{1}{2}\nu_{\mu}^{2}\sigma\frac{\partial^{2}G}{\partial\mu^{2}}+\frac{1}{2}\nu_{\sigma}^{2}\sigma \frac{\partial^{2}G}{\partial\sigma^{2}}
\end{align*}


\subsubsection{Log linearization method}
We guess $G = e^{A_0+ A_1\mu_t + A_2 \sigma_t}$ and linearize the non linear term in $G$ 
$$G^{-\frac{1}{\theta}} \approx \overline{G}(1-\frac{1}{\theta}(A_1(\mu_t -\overline{\mu}) + A_2 (\sigma_t-1))$$
We can then solve for $A_0, A_1, A_2$ by setting the constant term, the term in $\mu_t$ and the term in $\sigma_t$ to zero in the linearized PDE. We obtain
\begin{align*}
	0&=-\rho  \overline{G}A_1 + 1-\gamma -\kappa_\mu A_1\\\
	0&=-\rho  \overline{G}A_2 -\frac{1}{2}(1-\gamma)\gamma \nu_D^2 + \kappa_\sigma (1-\sigma)A_2 + \frac{1}{2}\nu_\mu^2A_1^2 +\frac{1}{2}\nu_\sigma^2A_2^2
\end{align*}

Bansal Yaron (2004) find
\begin{align*}
	A1 &= \theta \frac{1-\frac{1}{\psi}}{1-0.997 e^{-\kappa_\mu}}\\
	A2 &= 0.5\theta^2\frac{(1 - \frac{1}{\psi})^2 + (A_1  0.997 \frac{\nu_\mu}{\nu_D})^2}{1-0.997e^{-\kappa_\sigma}}
\end{align*}
\subsubsection{Finite Difference Method}
We can discretize this PDE on a grid using a Finite Difference Scheme. 
\begin{itemize}
	\item We upwind the first derivative, i.e. we approximate $\partial G$ by a forward difference when the drift is positive and a backward difference when the drift is negative. 
	\item At the border of the grid, the PDE involves the value of $G$ outside the grid (through the second derivative). To get the value of $G$ at these nodes, we apply a boundary counditions: state variables have reflecting boundaries at the borders. This gives a supplementary condition of the form $\partial G = 0$ at the borders\footnote{At the frontier we have $$dG_{t} = G'(x) \sigma(x) \sqrt{dt} + G'(x)\mu(x) dt + \frac{1}{2}G''(x)\sigma^2(x)dt + o(dt)$$
	Since $dG_t$ is $O(dt)$, we must have $G'(x) = 0$, at least when $\sigma(x) \neq 0$.At the points $j = 0$, the volatility is zero. 
	At the frontier $\sigma = 0$, the term involving the second derivative disappears naturally so we don't need to add a reflecting boundary conditon}
\end{itemize}
To handle boundary conditions, 
To sum up the scheme is 
\begin{align*}
	0&= \rho \theta[(G_{ij})^{1-\frac{1}{\theta}}- G_{ij}]+G_{ij}((1-\gamma)\mu_i-\frac{1}{2}(1-\gamma)\gamma\nu_D^2\sigma_j)\\
	&+(\kappa_{\mu_i}(\bar{\mu}-\mu_i))^+\frac{G_{i+1, j}-G_{i, j}}{\Delta \mu}+(\kappa_{\mu_i}(\bar{\mu}-\mu_i))^-\frac{G_{i, j}-G_{i-1, j}}{\Delta \mu}\\
	&+(\kappa_{\sigma_j}(1-\sigma_j))^+\frac{G_{i, j+1}-G_{i,j}}{\Delta \sigma}+(\kappa_{\sigma_j}(1-\sigma_j))^-\frac{G_{i, j}-G_{i,j-1}}{\Delta \sigma}\\
	&+\frac{1}{2}\nu_{\mu_i}^{2}\sigma_j\frac{G_{i+1, j} - 2 G_{i, j} + G_{i-1, j}}{(\Delta\mu)^2}+\frac{1}{2}\nu_{\sigma_j}^{2}\sigma_j\frac{G_{i, j+1} - 2 G_{i, j} + G_{i, j-1}}{(\Delta\sigma)^2}
\end{align*}
Denote $Y$ the vector of  $(G_{ij})_{1 \leq i,j\leq n}$. The scheme  defines a function $F$ such that $F(Y) = 0$. We can solve for $Y$ using one of these methods:
\begin{enumerate}
	\item Use a non linear solver for the system $F(Y) = 0$. These algorithms start with an initial guess, and update based on the Jacobian of $F$. In some PDE (not this one), you need a good initial guess. A technique is to solve the PDE for $\theta = 1$, and use the solution at an initial guess for other values of $\theta$.
	\item Use an ODE solver for the system $F(Y) = \dot{Y}$. The solution when $T\rightarrow +\infty$ is the solution of the PDE. This method is called ``the method of lines''.
\end{enumerate}
Both methods require the jacobian of $F$, which can generally be automatically computed using automatic or numerical differenciation. 
A solution that would \textit{not} work is to solve for $G$ by iterating over time
\begin{align*}
	\frac{G_{n+1}-G_n}{\Delta t} &= F(G_n)
\end{align*}
This method can be seen as a special case of a non linear solver (fixed point method) or as a special case of an ODE solver (in this context, it is called the Euler method). The criterion for the convergence of this method is that $F$ is monotonous in $G$ (Barles Souganadis theorem). This is not the case here, due to the non linear term  $\rho \theta[(G_{ij})^{1-\frac{1}{\theta}}- G_{ij}]$.

\subsubsection{Spectral Method (= collocation method)}
We can also solve the PDE by looking at solutions of the form
$$V(\mu, \sigma) = \sum_{kl} a_{kl} \phi_k(\mu)\psi_l(\sigma)$$
where $\phi, \psi$ denote any basis of functions (splines or chebyshev polynomials).
The value function is characterized by its coordinates $a_{kl}$ on this basis. Writing the PDE on a grid gives a non linear system in term of the coordinates, which we can solve. We use the same border condition and initial guess as the Finite Difference method.

\subsection{Comparison}

\begin{tabular}{|c|c|c|c|}
	\hline 
	\multicolumn{4}{c}{Correspondances}
	\\
	\hline 
	\hline 
	mean growth rate & $\mu$ & $\bar{\mu}$ & $\mu=\bar{\mu}$
	\\
	\hline 
	mean volatility & $\sigma^{2}$ & $\nu_{D}$ & $\sqrt{\sigma^{2}}=\nu_{D}$
	\\
	\hline 
	growth persistence & $\rho$ & $\kappa_{\mu}$ & $ - \log(\rho) = \kappa_\mu$ 
	\\
	\hline 
	volatility persistence & $\nu_{1}$ & $\kappa_{\sigma}$ & $-\log\left(\nu_{1}\right)=\kappa_{\sigma}$
	\\
	\hline 
	growth rate volatility & $\varphi_{e}$ & $\nu_{\mu}$ & $\varphi_{e}\times\sqrt{\sigma^{2}}=\nu_{\mu}$
	\\
	\hline 
	volatility volatility & $\sigma_{w}$ & $\nu_{\sigma}$ & $\sigma_{w}/\sigma^{2}=\nu_{\sigma}$
	\\
	\hline 
	time discount & $\delta$ & $\rho$ & $-\log\left(\delta\right)=\rho$\\
	\hline
\end{tabular}


\begin{tabular}{|c|c|c|}
	\hline 
	\multicolumn{3}{|c|}{BY 2004}
	\\
	\hline 
	Name & Discrete Time & Continuous Time
	\\
	\hline 
	mean growth rate & $\mu =  0.0015$ &  $\overline{\mu} = 0.0015$
	\\
	\hline 
	mean volatility & $\sigma = 0.0078$ & $\nu_{D}=0.0078$
	\\
	\hline 
	growth persistence & $\rho = 0.979$ & $\kappa_{\mu}= 0.0212 $
	\\
	\hline 
	volatility persistence & $\nu_1 = 0.987$ & $\kappa_{\sigma}=0.0131$ 
	\\
	\hline 
	growth rate volatility &  $\phi_e = 0.044$ & $\nu_{\mu}= 0.0003432$
	\\
	\hline 
	volatility volatility & $\sigma_w = 0.0000023$ & $\nu_{\sigma}=0.0378$ 
	\\
	\hline 
	time discount & $\delta = 0.998$  & $\rho=0.002$
	\\
	\hline
\end{tabular}

\begin{tabular}{|c|c|c|}
	\hline 
	\multicolumn{3}{|c|}{BKY 2007}
	\\
	\hline
	Name & Discrete Time & Continuous Time 
	\\
	\hline 
	\hline 
	mean growth rate & $\mu =  0.0015$ &  $\overline{\mu} = 0.0015$
	\\
	\hline 
	mean volatility & $\sigma = 0.0072$ & $\nu_{D}=0.0072$
	\\
	\hline 
	growth persistence & $\rho = 0.975$ & $\kappa_{\mu}= 0.0253 $
	\\
	\hline 
	volatility persistence & $\nu_1 = 0.999$ & $\kappa_{\sigma}=0.001$ 
	\\
	\hline 
	growth rate volatility &  $\phi_e = 0.038$ & $\nu_{\mu}= 0.00274$
	\\
	\hline 
	volatility volatility & $\sigma_w = 0.00000283$ & $\nu_{\sigma}=0.05401$ 
	\\
	\hline 
	time discount & $\delta = 0.9989$  & $\rho=0.0011$
	\\
	\hline
\end{tabular}




\end{document}