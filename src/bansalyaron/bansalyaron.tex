\documentclass[english]{article}
\usepackage[T1]{fontenc}
\usepackage[latin9]{inputenc}
\usepackage{color}
\usepackage{amsmath}
\usepackage{amssymb}
\usepackage{esint}
\usepackage{babel}
\begin{document}

\section{Derivation}
HJB is
$$0 = \rho \theta (\frac{C_t^{\frac{1-\gamma}{\theta}}}{((1-\gamma)V_t)^{\frac{1}{\theta}}}-1)dt + E[\frac{dV_t}{V_t}]$$
Define $G_t$ such that
$$V_t = \frac{C_t^{1-\gamma}}{1-\gamma}\rho^\theta G_t^\theta$$
Denote $\mu_{C}$ and $\sigma_{C}$ the geometric drift of $C_t$, and $\mu_G, \sigma_G$ the geometric drift and volatility of $G_t$ and apply Ito.
$$0 = \frac{\theta}{G} - \theta\rho + (1-\gamma)\mu_{C}  + \theta\mu_G - \frac{1}{2}(1-\gamma)\gamma\sigma_{C}'\sigma_{C} + \theta(1-\gamma)\sigma_G'\sigma_{C} + \frac{1}{2}\theta(\theta-1)\sigma_G^2$$
By taking FOC for consumption, one can check that $G$ corresponds to the wealth ratio

\section{Long run risk model}

\subsection{Derivation}
We now assume that the evolution of consumption is driven by two state variables $\mu_{t}$ and $\sigma_{t}$:
\begin{align*}
	\frac{dC_{t}}{C_{t}} & =  \mu_{t}dt+\nu_{D}\sqrt{\sigma_{t}}dZ_{t}\\
	d\mu_{t} & =  \kappa_{\mu}(\bar{\mu}-\mu_{t})dt+\nu_{\mu}\sqrt{\sigma_{t}}dZ_{t}^{\mu}\\
	d\sigma_{t} & =  \kappa_{\sigma}(1-\sigma_{t})dt+\nu_{\sigma}dZ_{t}^{\sigma}
\end{align*}
We write $G_t = G(\mu, \sigma)$ and we get the PDE
\begin{align*}
	0&= \frac{\theta}{G} - \rho \theta +  (1-\gamma)(\mu - \frac{1}{2}\gamma\nu_D^2\sigma)\\
	&+  \kappa_{\mu}\theta(\bar{\mu}-\mu)\frac{\partial_\mu G}{G}+  \kappa_{\sigma}\theta(1-\sigma)\frac{\partial_\sigma G}{G}\\
	&+\frac{1}{2}\nu^2_\mu\sigma(\theta\frac{\partial^{2}_\mu G}{G}+ \theta(\theta-1)(\frac{\partial_\mu G}{G})^2) + \frac{1}{2}\nu^2_\sigma(\theta\frac{\partial^{2}_\sigma G}{G}+ \theta(\theta-1)(\frac{\partial_\sigma G}{G})^2)
\end{align*}

\subsection{Numerical Methods}
We can use two methods

\subsubsection{Finite Difference Method}
We can discretize this PDE on a grid using a Finite Difference Scheme. 
\begin{itemize}
	\item We upwind the first derivative, i.e. we approximate $\partial_x G$ by a forward difference when the drift of the $x$ variable is positive and a backward difference when the drift is negative. 
	\item At the border of the grid, the PDE involves the value of $G$ outside the grid through the second derivative. To get the value of $G$ at these nodes, we apply a boundary counditions: state variables have reflecting boundaries at the borders. This gives a supplementary condition of the form $\partial G = 0$ at the borders\footnote{At the frontier we have $$dG_{t} = G'(x) \sigma(x) \sqrt{dt} + G'(x)\mu(x) dt + \frac{1}{2}G''(x)\sigma^2(x)dt + o(dt)$$
	Since $dG_t$ is $O(dt)$, we must have $G'(x) = 0$, at least when $\sigma(x) \neq 0$.At the points $j = 0$, the volatility is zero. 
	At the frontier $\sigma = 0$, the term involving the second derivative disappears naturally so we don't need to add a reflecting boundary conditon}
\end{itemize}
To sum up the scheme is 
\begin{align*}
	\Delta_\mu G &= (\mu_i \leq \bar{\mu})\frac{G_{i+1, j}-G_{i, j}}{\Delta \mu}+ \mu_i > \bar{\mu})\frac{G_{i, j}-G_{i-1, j}}{\Delta \mu}\\
	\Delta^2_\mu G &= \frac{G_{i+1, j} + G_{i-1, j} - 2 G_{i,j}}{(\Delta \mu)^2}\\
	\Delta_\sigma G &= (\sigma_j \leq 1)\frac{G_{i+1, j}-G_{i, j}}{G_{i,j}\Delta \mu}+ (\sigma_j > 1)\frac{G_{i, j}-G_{i-1, j}}{G_{i,j}\Delta \mu}\\
	\Delta^2_\sigma G &= \frac{G_{i, j+1} + G_{i, j-1} - 2 G_{i,j}}{(\Delta \sigma)^2}\\
	0&= \frac{\theta}{G_{i,j}} - \rho \theta + (1-\gamma)\mu_i -\frac{1}{2}(1-\gamma)\gamma\nu_D^2\sigma_j\\
	&+\theta\kappa_{\mu_i}(\bar{\mu}-\mu_i) \frac{\Delta_\mu G}{G_{ij}}+ \theta\kappa_{\sigma_i}(1-\sigma_j)\frac{\Delta_\sigma G}{G_{ij}}\\
	&+\frac{\theta}{2}\nu^2_{\mu_i}\sigma_j(\frac{\Delta^2_\mu G}{G_{ij}} + (\theta-1)(\frac{\Delta_\mu G}{G_{ij}})^2) + \frac{\theta}{2}\nu^2_{\sigma_j}(\frac{\Delta^2_\sigma G}{G_{ij}} + (\theta-1)(\frac{\Delta_\sigma G}{G_{ij}})^2)
\end{align*}
Denote $Y$ the vector of  $(G_{ij})_{1 \leq i,j\leq n}$. The scheme  defines a function $F$ such that $F(Y) = 0$. We can solve for $Y$ using one of these methods:
\begin{enumerate}
	\item Use a non linear solver for the system $F(Y) = 0$. These algorithms start with an initial guess, and update based on the Jacobian of $F$. In some PDE (not this one), you need a good initial guess. A technique is to solve the PDE for $\theta = 1$, and use the solution at an initial guess for other values of $\theta$.
	\item Use an ODE solver for the system $F(Y) = \dot{Y}$. The solution when $T\rightarrow +\infty$ is the solution of the PDE. This method is called ``the method of lines''.
	\item 
\end{enumerate}
Both methods require the jacobian of $F$, which can generally be automatically computed using automatic or numerical differenciation. 
A less robust method solves $G$ by iterating over time
\begin{align*}
	\frac{G_{n+1}-G_n}{\Delta t} &= F(G_n)
\end{align*}
This method can be seen as a special case of a non linear solver (fixed point method) or as a special case of an ODE solver (i.e. Euler method). The criterion for the convergence of this method is that $F_{ij}$ is decreasing in $G_{i,j}$ (Barles Souganadis theorem). This is not the case if $\theta < 0$ (which the relevant case in BKY 2007 with $\gamma = 7.5$ and $\psi = 1.5$)

\subsubsection{Spectral Method (= collocation method)}
We can also solve the PDE by looking at solutions of the form
$$V(\mu, \sigma) = \sum_{kl} a_{kl} \phi_k(\mu)\psi_l(\sigma)$$
where $\phi, \psi$ denote any basis of functions (splines or chebyshev polynomials).
The value function is characterized by its coordinates $a_{kl}$ on this basis. Writing the PDE on a grid gives a non linear system in term of the coordinates, which we can solve. We use the same border condition and initial guess as the Finite Difference method.

\subsection{Comparison}
\begin{tabular}{|c|c|}
	\hline 
	\multicolumn{2}{|c|}{Correspondences}
	\\
	\hline
	\hline
	$\Delta c_{t+1} = \mu + x_t + \sigma_t \eta_{t+1}$ & 		$\frac{dC_{t}}{C_{t}}  =  \mu_{t}dt+\nu_{D}\sqrt{\sigma_{t}}dZ_{t}$
	\\
	\hline
	$x_{t+1} =  \rho x_t + \phi_e \sigma_t e_{t+1}$ & 		$d\mu_{t}  =  \kappa_{\mu}(\bar{\mu}-\mu_{t})dt+\nu_{\mu}\sqrt{\sigma_{t}}dZ_{t}^{\mu}$
	\\
	\hline
	$\sigma_{t+1}^2 = \sigma^2 + \nu_1 (\sigma_t^2 - \sigma^2) + \sigma_w w_{t+1}$ & 		$d\sigma_{t}  =  \kappa_{\sigma}(1-\sigma_{t})dt+\nu_{\sigma}dZ_{t}^{\sigma}$
	\\
	\hline
\end{tabular}
\\
\begin{tabular}{|c|c|c|}
	\hline 
	\multicolumn{3}{|c|}{Correspondences}
	\\
	\hline 
	\hline 
	growth rate & $\mu$ & $\bar{\mu}= \mu\times 12$
	\\
	\hline 
	volatility & $\sigma^{2}$ & $\nu_{D} = \sigma\times\sqrt{12}$
	\\
	\hline 
	growth persistence & $\rho$ & $\kappa_{\mu} = (1 - \rho)\times 12$ 
	\\
	\hline 
	volatility persistence & $\nu_{1}$ & $\kappa_{\sigma}= (1-\nu_1) \times 12$
	\\
	\hline 
	growth rate volatility & $\varphi_{e}$ & $\nu_{\mu} = (\varphi_{e}\sigma 12)\times\sqrt{12}$
	\\
	\hline 
	volatility volatility & $\sigma_{w}$ & $\nu_{\sigma} = (\sigma_{w} / \overline{\sigma}^2)\times\sqrt{12}$
	\\
	\hline 
	time discount & $\delta$ & $\rho = (1-\delta)\times12$
	\\
	\hline
\end{tabular}
\\
\begin{tabular}{|c|c|c|}
	\hline 
	\multicolumn{3}{|c|}{BY 2004}
	\\
	\hline 
	Name & Discrete Time & Continuous Time
	\\
	\hline 
	growth rate & $\mu =  0.0015$ &  $\overline{\mu} = 0.018$
	\\
	\hline 
	volatility & $\sigma = 0.0078$ & $\nu_{D}=0.027$
	\\
	\hline 
	growth persistence & $\rho = 0.979$ & $\kappa_{\mu}= 0.252$
	\\
	\hline 
	volatility persistence & $\nu_1 = 0.987$ & $\kappa_{\sigma}=0.156$ 
	\\
	\hline 
	growth rate volatility &  $\phi_e = 0.044$ & $\nu_{\mu}= 0.0143$
	\\
	\hline 
	volatility volatility & $\sigma_w = 0.0000023$ & $\nu_{\sigma}=0.131$ 
	\\
	\hline 
	time discount & $\delta = 0.998$  & $\rho=0.024$
	\\
	\hline
\end{tabular}
\\
\begin{tabular}{|c|c|c|}
	\hline 
	\multicolumn{3}{|c|}{BKY 2007}
	\\
	\hline
	Name & Discrete Time & Continuous Time 
	\\
	\hline 
	\hline 
	mean growth rate & $\mu =  0.0015$ &  $\overline{\mu} = 0.018$
	\\
	\hline 
	mean volatility & $\sigma = 0.0072$ & $\nu_{D}=0.025$
	\\
	\hline 
	growth persistence & $\rho = 0.975$ & $\kappa_{\mu}= 0.3$
	\\
	\hline 
	volatility persistence & $\nu_1 = 0.999$ & $\kappa_{\sigma}=0.012$ 
	\\
	\hline 
	growth rate volatility &  $\phi_e = 0.038$ & $\nu_{\mu}= 0.0114$
	\\
	\hline 
	volatility volatility & $\sigma_w = 0.00000283$ & $\nu_{\sigma}=0.189$ 
	\\
	\hline 
	time discount & $\delta = 0.9989$  & $\rho=0.0132$
	\\
	\hline
\end{tabular}

\section{Loglinearization}
\begin{itemize}
	\item 
	\begin{align*}
		0&= \theta e^{-\overline{g}}(1 -A_1(\mu-\overline{\mu}) -A_2(\sigma - 1)) - \rho \theta +  (1-\gamma)\mu - \frac{1}{2}(1-\gamma)\gamma\nu_D^2\sigma\\
		&+ \theta \kappa_{\mu}(\bar{\mu}-\mu)A_1+  \theta\kappa_{\sigma}(1-\sigma)A_2\\
		&+\frac{\theta^2}{2}\nu^2_\mu\sigma A_1^2 + \frac{\theta}{2}\nu^2_\sigma A_2^2
	\end{align*}
	We obtain
	\begin{align*}
		A_1&= \frac{1-\frac{1}{\psi}}{e^{-\overline{g}} + \kappa_\mu}\\
		A_2 &=\frac{1}{2} \frac{(\theta-\frac{\theta}{\psi})(\theta-\frac{\theta}{\psi}-1) \nu_D^2+ A_1^2\theta^2\nu_\mu^2 }{\theta(e^{-\overline{g}} + \kappa_\sigma)}
	\end{align*}
	BY obtains
	\begin{align*}
		A_1&= \frac{1-\frac{1}{\psi}}{1 - \frac{1-\kappa_\mu}{1 + e^{-\overline{g}}}}\\
		A_2&= \frac{1}{2}\frac{(\theta-\frac{\theta}{\psi})^2\nu_D^2 + A_1^2 \theta^2(\frac{e^{-\overline{g}}}{1 +  e^{-\overline{g}}})^2 \nu_\mu^2}{\theta(1 - \frac{e^{-\overline{g}}}{1 +  e^{-\overline{g}}}(1-\kappa_\sigma))}
	\end{align*}
	\item The HJB equation can  be obtained by applying Euler equation to the return of the wealth portfolio
	\begin{align*}
		\frac{EdR}{dt}  &= \frac{1}{G} + \mu_G+\mu_C + \sigma_G'\sigma_C\\
		\sigma_{dR} &= \sigma_G + \sigma_C\\
		r&= \theta\rho +\frac{\theta}{\psi} \mu_C + (1-\theta) EdR_C + \frac{1}{2}(\frac{\theta}{\psi}-1)\sigma^2_C -\frac{1}{2}\theta \sigma^2_{R_c} + \frac{\theta}{\psi}(1-\theta)\sigma_C\sigma_{dR_C}\\
		\kappa&= \frac{\theta}{\psi}\sigma_C +(1-\theta)\sigma_{dR}
	\end{align*}
	We would obtain the same than Bansal Yaron by transforming the return:
	\begin{align*}
		R_{t+1} &= \frac{C_{t+1}+ W_{t+1}}{W_t}\\
		&= \frac{C_{t}}{W_t}(\frac{C_{t+1}}{C_t}+\frac{W_{t+1}}{C_t})\\
		&= \frac{1}{G_t}\frac{C_{t+1}}{C_t}(1+G_{t+1})
	\end{align*}
	\begin{align*}
		d\ln R &= \kappa_0dt + \kappa_1(d\ln G + \ln G dt) - \ln Gdt + d\ln C
	\end{align*}
	\item In continuous time the dividend is known at $t$
	\begin{align*}
		R_{t+1} &= \frac{C_{t}+ W_{t+1}}{W_t}\\
		&= \frac{C_{t}}{W_t}(1 + \frac{C_{t+1}}{C_{t}}\frac{W_{t+1}}{C_{t+1}})\\
		&= \frac{1}{G_t}(1 + \frac{C_{t+1}}{C_t}G_{t+1})
	\end{align*}
	\begin{align*}
		d\ln R &= \kappa_0dt + \kappa_1(d\ln G + \ln G dt + d\ln C) - \ln Gdt
	\end{align*}
\end{itemize}
\end{document}